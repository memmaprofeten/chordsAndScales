%! Author = joel
%! Date = 14.6.2020

% Preamble
\documentclass[11pt]{book}

% Packages
\usepackage{amsmath}
\usepackage{guitarchordschemes}
\usepackage{guitar}
\title{My chords, scales and songs that I have learned}
\author{Joel Sågfors}
\date{\today}
% Document
\makeindex
\begin{document}
    \maketitle


    \chapter{Chords}
    Learning chords is a cornerstone to learning guitar. Strumming chords is usually the first way a beginner plays the guitar.
    In chapter we explore the different chords for each note as I learn them, starting from the open major chord.


    \section{A chord}
    Starting from the first letter in the alphabet, the A chord is also one of the first chords that a guitar player learns,
    and is used in a lot of songs.

    \subsection{Open A major}
    This chord is easier to play with the fingers "out-of-order" as listed in the chord-box.
    The fingering makes it easier to get all fingers close to the fret.
    The root is in the fifth string.

    \chordscheme[
    name   = A,
    finger = {2/4:2,2/3:1,2/2:3},
    ring   = {5,1},
    mute   = {6}
    ]

    The notes in this chord is, A, E, A, C, E

    \subsection{A minor}
    Pretty self-explanatory, the minor version of A.

    \chordscheme[
    name = Am,
    finger = {2/4:2, 2/3:3, 2/2.1},
    ring = {5,1},
    mute = {6}
    ]

    The notes in this chord are A, E, A, B\#, E.


    \section{F chord}
    The second to last note in the note circle, and a dreaded chord for many beginners, because the 'normal' version of
    F is barre.
    With proper practice you can however learn the barred version and in the process you learned a lot of chords as this shape can be moved around.

    \subsection{Barre F}

    \chordscheme[
    name = F,
    finger = {3/5:4,3/4:4,2/3:2},
    barre = {1/1-6:1}
    ]


    \chapter{Scales}


    \section{Minor Pentatonic}
    One of, if not the most common scales in rock and blues music is the minor pentatonic scale.
    The name means minor five note scale and that is pretty much what it is.
    The beauty of the minor pentatonic scale is that it can be played anywhere on the guitar neck, which makes it very versatile.

    The basis of the scale goes as follows:

    \bigskip

    \scales[
    name = Minor Pentatonic,
    finger = {
    2/1:1, 5/1:4,
    2/2:1, 5/2:4,
    2/3:1, 4/3:3,
    2/4:1, 4/4:3,
    2/5:1, 4/5:3,
    2/6:1, 5/6:4
    },
    root = {2/6, 4/4, 2/1}
    ]


    \section{Major Scales}
    Major scales are for the major notes in the circle.
    I learned this is primary school but has since needed to relearn

    \subsection{C Major}
    The first major scale I learned for guitar was C Major.
    It consists of only major notes and can be played from the top of the neck, where I learned.

    \scales[
    name = C Major
    position = 1,
    finger = {
    1/1:1,3/1:3,
    1/2:1,3/2:3,
    2/3:2,
    2/4:2,3/4:3,
    2/5:2,3/5:3,
    1/6:2,3/6:3
    }
    ]

    \subsubsection{Chords in C Major}
    These are the chords in C:

    C, Dm, Em, F, G, Am

    Numbered like.

    1 = C
    2 = Dm
    3 = Em
    4 = F
    5 = G
    6 = Am


    \chapter{Songs}


    \section{Dreams - Fleetwood Mac}
    Dreams by Fleetwood Mac is a nice chill song about some players, not sure what they are playing though.
    I will shortly present how to play the parts of the song and also do and chord overlay for the lyrics I think.

    \begin{guitar}
    \end{guitar}
    \chapter*{Bonus Chapter: Riffs}


\end{document}